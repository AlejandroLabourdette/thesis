\begin{opinion}
El trabajo ``Lógica para un Gestor de Certificados Académicos'' desarrollado por el estudiante José Alejandro Labourdette - Lartigue Soto, cumple con los requisitos para la culminación de la carrera de Ciencia de la Computación de la Universidad de La Habana.

El trabajo forma parte de una solución descentralizada para un sistema de certificación de títulos académicos basado en la tecnología de blockchain Hyperledger Fabric.

Siendo Hyperledger Fabric (HF) una de las plataformas usadas para el desarrollo de nuestras aplicaciones blockchain y siendo una debilidad la ausencia de mecanismos automatizados y confiables para dar el servicio de certificación de titulaciones por parte de las instituciones de educación, hacen pertinente y útil el desarrollo de tal solución.

El diplomante ha demostrado interés y dedicación por el tema, cumpliendo con todos los requisitos definidos por el cliente. También es válido aclarar que ha mostrado muy buenas habilidades técnicas en el desarrollo de la solución. Para ello, comenzó con la asimilación y estudio de las tecnologías indicadas por los tutores, mostrando además buenas capacidades de asimilación e independencia.

Exhortamos al estudiante a continuar colaborando con nuestro colectivo y su repositorio para darle continuidad a este trabajo y llevar los resultados a un servicio real de aplicación cuando se contraten con los usuarios que se beneficiarían del mismo.

Por tanto, felicitamos al estudiante por la labor desarrollada y consideramos que la tesis reúne los estándares metodológicos exigidos por la Facultad de Matemática y Computación de la Universidad de la Habana, para ser presentada y sometida a evaluación en su ejercicio de defensa.

\:

La Habana, Diciembre 7 de 2022

\begingroup
\wildcard{Ing. Daniel Frias Mena}
\hspace{0.1cm}
\wildcard{MSc. Camilo Denis González}
\hspace{0.1cm}
\wildcard{Dr. Miguel Katrib Mora}
\par
\endgroup
    
\end{opinion}
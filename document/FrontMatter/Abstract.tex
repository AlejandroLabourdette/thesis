\begin{resumen}
	Los certificados académicos son documentos que demuestran que una persona tiene conocimientos y herramientas para desempeñarse en una determinada profesión o perfil. Emitir y verificar estas certificaciones para usarla en determinados procesos como pueden ser solicitudes de estudios superiores, reclutamiento laboral, etc; requiere por lo general de un procedimiento burocrático que puede ser demorado. La aplicación de la tecnología blockchain a los protocolos de verificación de certificados a través de una arquitectura integral brindaría autenticidad y podría simplificar el proceso reduciendo el tiempo que consume. En este trabajo de diploma de fin de carrera se propone un sistema automatizado para gestionar y consultar las certificaciones de estudiantes manteniendo su control, propiedad y veracidad. Este sistema se ha desarrollado usando Hyperledger Fabric aprovechando las ventajas de la tecnología blockchain para compartir y verificar la certificación electrónica. Su aplicación en las universidades facilitaría el proceso de emisión, intercambio, verificación e invulnerabilidad de estas certificaciones.
	
	\:
	
	\textbf{Palabras claves}: Blockchain, Certificados Académicos, Sistema Distribuido, Hyperledger Fabric, Gestor Universitario
	
\end{resumen}

\begin{abstract}
	
	Academic certificates are documents that demonstrate a person knowledge and tools to perform in a certain profession or profile. Issue and verify these certifications to be used in certain processes such as applications for higher education, job recruitment, etc.; generally requires a bureaucratic procedure that can be delayed. The application of blockchain technology to certificate verification protocols through a comprehensive architecture would provide authenticity and could simplify the process reducing the time it consumes. In this final degree diploma project, an automated system is proposed to manage and consult student certifications, maintaining their control, ownership and veracity. This system has been developed using Hyperledger Fabric, taking advantage of blockchain technology to share and verify electronic certification. Its application in universities would facilitate the process of issuance, exchange, verification and invulnerability of these certifications.
	
	\:
	
	\textbf{Keywords}: Blockchain, Academic Certificates, Distributed System, Hyperledger Fabric, University Manager
\end{abstract}
\begin{resumen}
	Los certificados son documentos que prueban capacidad para realizar determinadas tareas, en el caso particular de los certificados académicos demuestran que el estudiante tiene conocimientos y herramientas para desempeñarse en determinado perfil. 
	Emitir y verificar estas certificaciones para procesos como pueden ser: solicitudes de estudios superiores,  reclutamiento laboral, etc; requiere muchos pasos que demoran días en completarse.
	La aplicación de la tecnología blockchain a los protocolos de verificación de certificados a través de una arquitectura integral brinda autenticidad y reduce significativamente dicho tiempo.
	En este documento, se ha propuesto un sistema automatizado para gestionar y consultar las certificaciones de estudiantes mientras se mantiene su control y propiedad.
	Este sistema está desarrollado usando Hyperledger Fabric y aprovecha las ventajas de la tecnología blockchain para compartir y verificar la certificación electrónica.
	Su aplicación en las universidades proporcionará una baja latencia para la emisión, el intercambio y la verificación de estas certificaciones.
	
	Palabras claves: Blockchain, Certificados Académicos, Sistema Distribuido, Hyperledger Fabric, Gestor Universitario
\end{resumen}

\begin{abstract}
	Certificates are documents that prove the ability to perform certain tasks, in the particular case of academic certificates, they show that the student has the knowledge and tools to perform in a certain profile.
	Issue and verify these certifications for processes such as: applications for higher education, job recruitment, etc; requires many steps that take days to complete and is considered time consuming.
	The application of blockchain technology to certificate verification protocols through a comprehensive architecture provides authenticity and significantly reduces said time.
	In this document, an automated system has been proposed to manage and consult student certifications while maintaining their control and ownership.
	This system is developed using Hyperledger Fabric and takes advantage of blockchain technology to share and verify the certificates.
	Its application in universities will provide low latency for the issuance, exchange and verification of these certifications.
	
	Keywords: Blockchain, Academic Certificates, Distributed System, Hyperledger Fabric, University Manager
\end{abstract}
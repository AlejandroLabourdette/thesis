\begin{recomendations}
El sistema propuesto en este documento tiene áreas en las que seguir mejorando y una de las formas más comunes de mejorar es mediante la retroalimentación. Una vez las autoridades pertinentes se encuentren utilizando el sistema se les puede consultar que funcionalidades creen que son necesarias agregar o eliminar de las ya existentes. Deben ser consultados también las opiniones de los desarrolladores de terceras aplicaciones que decidan conectarse a los endpoints del sistema.

Se recomienda que en una futura versión el sistema de usuarios esté también escrito en la blockchain y se empiecen a utilizar certificados digitales (no confundir con los certificados académicos sobre los que trataba este trabajo) para cada uno de esos usuarios. Para ello es necesario el uso de una Autoridad Certificadora que gestione el sistema de llaves públicas y privadas que hace posible la criptografía necesaria para garantizar la seguridad. Con estos certificados digitales, los usuarios que interactúan y gestionan el sistema pueden dar fe de determinado criterio usando su firma digital.


Una vez concluido el proceso de desarrollo de la solución propuesta, cumplidos los objetivos trazados y teniendo en cuenta que es una primera versión del sistema, se recomienda:
%aprovechando sobre todo la escalabilidad de la cual se ha intentado dotar al proyecto:
\begin{itemize}
  	\item Ampliar el equipo con un nuevo diseñador de interfaces de usuario encargado de crear interfaces mucho más profesionales, consiguiendo mejorar de manera notoria la experiencia del usuario.
   	%\item Someter el sistema a pruebas de calidad.
   	\item Analizar los beneficios que traería el desarrollo de esta solución como una aplicación móvil y el empleo de códigos QR.
    \item Permitir el acceso al \textit{software} a otras universidades del país con el objetivo de crear un consorcio de instituciones académicas para la validación y emisión de certificaciones.
\end{itemize}
    
\end{recomendations}

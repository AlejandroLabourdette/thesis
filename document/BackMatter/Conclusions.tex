\begin{conclusions}
Los certificados académicos generados por instituciones cubanas suelen ser emitidos en papel u otro formato físico. Esto, a pesar de presentar ventajas como el total control por parte del destinatario sobre su certificado y la dificultad a la hora de realizar falsificaciones del mismo, hace engorrosos y lentos los procesos que requieran su uso.
    
El presente trabajo ha logrado crear la lógica de un sistema que, utilizando la tecnología blockchain con Hyperledger Fabric, gestiona de manera eficiente dichos certificados académicos, cumpliendo así con el objetivo general marcado. Guardar estas certificaciones usando blockchain las hace inmutables, privadas y válidas según el criterio de múltiples autoridades descentralizadas. El sistema propuesto permite realizar cientos de transacciones en pocos segundos y los contratos inteligentes unidos al hashing aseguran y controlan su despliegue. Se puede comprobar la validez de los títulos de un estudiante en solo segundos mientras que la forma actual tarda cerca de 21 días. Por último, pero no menos importante, la transformación digital de la emisión y verificación de certificaciones contribuye enormemente a revertir los cambios climáticos negativos al reducir el consumo de papel.

Debido a la arquitectura utilizada y la modularización de sus componentes, el sistema propuesto es fácilmente mantenible y extensible para futuras versiones. Sirve de base para un sistema futuro más ambicioso que compita directamente con las certificaciones tradicionales. El desacople con el que fue diseñada la lógica permite a aplicaciones que actualmente se encuentren en producción, y que puedan requerir sus servicios, realizar pedidos al sistema y lograr una latencia de respuesta casi nula.
\end{conclusions}
